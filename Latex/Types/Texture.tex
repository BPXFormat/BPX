\section{BPX Type: Texture ('T')}

\subsection{Overview}
The Texture BPX uses 'T' as the type byte of BPX Main Header. This type provides optimized and efficient texture storage for 3D rendering APIs.
\newline
Below is a table describing the different sections to be expected in a BPXT:
\bpxsectiontable
{
    PixelArray & 1 & Yes & Yes \\
}

\subsection{TypeExt}
Contains information about the texture encoding.
\bpxfieldtable
{
    Array & Unsigned & 16 & Size of texture array \\
    Width & Unsigned & 4 & Texture width \\
    Height & Unsigned & 4 & Texture height \\
    Format & Unsigned & 4 & Texture format \\
    Flags & Unsigned & 4 & Flags \\
    MipMap Level & Unsigned & 8 & Number of mip maps \\
    Reserved & Unspecified & 72 & Blank, always 0 \\
}
\begin{center}
    \begin{bytefield}[bitwidth=1.5em]{16}
        \bitheader{0-15} \\
        \begin{rightwordgroup}{128 bits}
            \bitbox{16}{Array} \\
            \bitbox{4}{Width} & \bitbox{4}{Height} & \bitbox{4}{Format} & \bitbox{4}{Flags} \\
            \bitbox{8}{MipMap Level} & \bitbox{8}{Reserved} \\
            \bitbox{16}{Reserved} \\
            \bitbox{16}{Reserved} \\
            \bitbox{16}{Reserved} \\
            \bitbox{16}{Reserved}
        \end{rightwordgroup}
    \end{bytefield}
\end{center}

\subsubsection{Width}
The expected width of all pixel arrays in power of two form ($2^{k+1}$px where k is the stored number).

\subsubsection{Height}
The expected height of all pixel arrays in power of two form ($2^{k+1}$px where k is the stored number).

\subsection{Analysis on texture size storage}
Certain rendering APIs require that the textures are aligned to a specific implementation defined number of pixels per row. This number, also called stride, is usualy 8 or a power of two greater than 8.\newline
By assuming any BPX encoded texture is encoded with power of twos instead of their actual resolution in pixels, we can eliminate the need, in cases where the implementation uses power of two strides, to run a padding alignment before presenting the texture to this implementation. This also allows to reduce the field size for storing texture size: instead of using 32 bits or 64 bits we can store a texture size in 8 bits and keep relatively large texture sizes, \textbf{optimizing both storage space and application load speed}. Indeed the maximum texture size in each direction allowed by BPX would be $2^{15 + 1} = 2^{16} = 65536$. Most rendering API implementations do not support such large texture sizes.\newline
From there one might say that we should then use less than 8 bits. However, using less than 8 bits for storing the entire size vector would conflict with hardware indexing as most hardware only indexes 8 bits bytes.

\subsubsection{Format}
Available formats:
\bpxtable{c|c|C}{Name & Value & Notes}
{
    RGB & 0x1 & Standard 8 bits 3 channel RGB format \\
    RGBA & 0x2 & 8 bits 4 channel RGB with transparency level \\
    GreyScale & 0x3 & Single 8 bits channel representing grey scale level \\
    Float & 0x4 & Single channel 32 bits float texture \\
}

\subsubsection{Flags}
Bit mask based flags (or flags together). Currently the only supported flags are:
\bpxtable{c|c|C}{Name & Value & Notes}
{
    Array & 0x1 & Indicates the texture should be interpreted as a texture array \\
    Compressed & 0x2 & Indicates the texture should be GPU compressed on load \\
    CubeMap & 0x4 & Indicates the texture should be interpreted as a CubeMap \\
}

\subsubsection{MipMap Level}
Number of mip maps \cite{MipMap} to auto generate when loading this texture.

\subsubsection{Array}
Number of textures for creating a texture array. If this value is 0 consider this is not a texture array.\newline
Ignore this value if neither of Array or CubeMap flags are set.

\subsection{PixelArray}
Texture data array encoded as described by texture format.\newline
If the \textit{Array} value in the header is greater than 0 then a series of \textit{Array} count texture data arrays are saved one after the other, encoded with respect to the texture format described by the header.\newline
In case the CubeMap flag is set, the \textit{Array} field in the header should be 6 and this section should contain 6 texture data arrays.