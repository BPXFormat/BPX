\section{BPX Type: Package ('P')}

\subsection{Overview}
The Package BPX is using 'P' as the type byte of BPX Main Header. This type provides asset packages using the BPX format.
\newline
Below is a table describing the different sections to be expected in a BPXP:
\begin{center}
    {
        \rowcolors{2}
        {red!15}
        {blue!15}
        \begin{tabular}{|c|c|c|c|}
            \hline
            \textbf{Name} & \textbf{Type} & \textbf{Required} & \textbf{Single Time} \\

            \hline\hline
            AssetRegistry & 1 & Yes & Yes \\
            File & 2 & Yes & No \\
            Strings & 3 & Yes & Yes \\
            \hline
        \end{tabular}
    }
\end{center}

\subsection{TypeExt}
Contains information about the package content.
\begin{center}
    {
        \rowcolors{2}
        {red!15}
        {blue!15}
        \begin{tabular}{|c|c|c|c|}
            \hline
            \textbf{Name} & \textbf{Type} & \textbf{Size} & \textbf{Notes} \\
    
            \hline\hline
            AssetCount & Unsigned & 32 & Number of assets \\
            CompatibilityFlags & Unsigned & 8 & Information about supported env. \\
            Reserved & Unspecified & 24 & Blank, always 0 \\
            \hline
        \end{tabular}
    }
\end{center}
\begin{center}
    \begin{bytefield}[bitwidth=1.2em]{32}
        \bitheader{0-31} \\
        \begin{rightwordgroup}{64 bits}
            \bitbox{32}{AssetCount} \\
            \bitbox{8}{CompatibilityFlags} & \bitbox{24}{Reserved}
        \end{rightwordgroup}
    \end{bytefield}
\end{center}

\subsubsection{AssetCount}
Number of assets to expect in the AssetRegistry section.

\subsubsection{CompatibilityFlags}
Bit mask based flags (or flags together). These flags are used to check if a given asset package can be used with the current configuration of hardware/driver/rendering implementation.\newline
Currently the only supported flags are:
\begin{center}
    {
        \rowcolors{2}
        {red!15}
        {blue!15}
        \begin{tabular}{|c|c|c|}
            \hline
            \textbf{Name} & \textbf{Value} & \textbf{Notes} \\
    
            \hline\hline
            DirectX & 0x1 & Indicates compatibility with DirectX \cite{DirectX} \\
            OpenGL & 0x2 & Indicates compatibility with OpenGL \cite{OpenGL} \\
            Vulkan & 0x4 & Indicates compatibility with Vulkan \cite{Vulkan} \\
            Metal & 0x8 & Indicates compatibility with Metal \cite{Metal} \\
            \hline
        \end{tabular}
    }
\end{center}

\subsection{AssetRegistry}
The AssetRegistry is an array of data structures which size is defined by the AssetCount field present in the TypeExt.\newline
Below is the data structure to expect as entry in the array:
\begin{center}
    {
        \rowcolors{2}
        {red!15}
        {blue!15}
        \begin{tabular}{|c|c|c|c|}
            \hline
            \textbf{Name} & \textbf{Type} & \textbf{Size} & \textbf{Notes} \\
    
            \hline\hline
            VirtualPath & Unsigned & 32 & Pointer in the Strings section \\
            FileId & Unsigned & 32 & File section index \\
            \hline
        \end{tabular}
    }
\end{center}
\begin{center}
    \begin{bytefield}[bitwidth=1.4em]{32}
        \bitheader{0-31} \\
        \bitbox{32}{VirtualPath} \\
        \bitbox{32}{FileId}
    \end{bytefield}
\end{center}
All indexes are given as positions in the Section Header Table.

\subsubsection{VirtualPath}
Asset engine path as a pointer to a null terminated string in the strings section.

\subsection{Strings}
The strings section contains a list of null-terminated strings to be referenced by start offset from other sections.
