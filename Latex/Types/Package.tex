\section{BPX Type: Package ('P')}

\subsection{Overview}
The Package BPX is using 'P' as the type byte of BPX Main Header. This type provides asset packages using the BPX format.
\newline
Below is a table describing the different sections to be expected in a BPXP:
\bpxsectiontable
{
    Data & 1 & Yes & No \\
    FileTable & 2 & Yes & Yes \\
    Metadata & 254 & No & Yes \\
    Strings & 255 & Yes & Yes \\
}
NOTE: The order of Data sections MUST be contiguous. A file is allowed to be divided in multiple sections, however a file header MUST NOT be divided.

\bpxnote{Revision 2 (SDK $\ge$ 3.0)}
{
	\begin{itemize}
		\item Only one file or the continuation of one is allowed per Data section.
		\item A Data section is not allowed to store file headers anymore, the FileTable section is here to store all file headers.
		\item An additional required FileTable section MUST only be generated in a BPX rev 2 or later.
		\item All data of a given file must be stored in a contiguous set of Data sections.
	\end{itemize}
}

\subsection{TypeExt}
Contains general information about the Package.
\bpxfieldtable
{
	Architecture & Unsigned & 8 & Target architecture \\
	Platform & Unsigned & 8 & Target platform \\
	Generator & Unsigned & 16 & Generator ID \\
	Reserved & Unsigned & 96 & Blank, always 0 \\
}
\begin{center}
	\begin{bytefield}[bitwidth=1.0em]{32}
		\bitheader{0-31} \\
		\begin{rightwordgroup}{128 bits}
			\bitbox{8}{Architecture} & \bitbox{8}{Platform} & \bitbox{16}{Generator} \\
			\bitbox{32}{Reserved} \\
			\bitbox{32}{Reserved} \\
			\bitbox{32}{Reserved}
		\end{rightwordgroup}
	\end{bytefield}
\end{center}

\subsubsection{Architecture}
8 bits of architecture stored as an enumeration. The current allowed values are:
\bpxtable{C|c}{Name & Value}
{
	x86\_64 & 0 \\
	aarch64 & 1 \\
	x86 & 2 \\
	armv7hl & 3 \\
	any & 4 \\
}

\subsubsection{Platform}
8 bits of platform stored as an enumeration. The current allowed values are:

\bpxtable{C|c}{Name & Value}
{
	Linux & 0 \\
	MacOS & 1 \\
	Windows & 2 \\
	Android & 3 \\
	Any & 4 \\
}

\subsubsection{Generator} \label{sssec:Generator}
16 bits of generator identification stored as 2 ASCII characters. The current known values are:

\bpxtable{C|c}{ASCII string & Notes}
{
	"PK" & Generated by FPKG \\
	"BP" & Generated by BlockProject 3D Engine \\
	"BD" & Generated by BPX debug tools \\
}
\textbf{The generator identification is only useful to know the expected structure of the optional Metadata section.}

\subsection{Data}
The Data section is used to store one or more files. Each file begins by a file header which is defined as:

\bpxfieldtable
{
    Size & Unsigned & 64 & File size \\
    Path & Unsigned & 32 & Pointer in the Strings section \\
}

\subsubsection{Size}
The total size of the file to read. When extracting a file which size is greater than the remaining bytes in the section just continue reading from next section as if the next section was simply a continuation of the current one.

\subsubsection{Path}
The path as a pointer to a null terminated string in the strings section. Used to identify relative extraction location and/or the virtual path of the asset in a rendering application.

\bpxnote{Revision 2 (SDK $\ge$ 3.0)}
{
	The Data section only stores one file or is the extension of the previous Data section. Additionally, a Data section MUST not contain any file header.

	\subsection{FileTable}
	The FileTable section stores an array of file headers where each file header points to an actual Data section where the start of the file content is expected.
	\newline
	A file header is represented by the following structure:
	
	\bpxfieldtable
	{
		Size & Unsigned & 64 & File size \\
		Path & Unsigned & 32 & Pointer in the Strings section \\
		Start & Unsigned & 32 & Index of the starting data section \\
	}

	\begin{center}
		\begin{bytefield}[bitwidth=0.69em]{64}
			\bitheader{0-63} \\
			\bitbox{64}{Size} \\
			\bitbox{32}{Path} \bitbox{32}{Start}
		\end{bytefield}
	\end{center}

	\subsubsection{Size}
	The total size of the file to read. When extracting a file which size is greater than the remaining bytes in the section just continue reading from next section as if the next section was simply a continuation of the current one.
	\newline
	\textbf{If the size of the file does not exceed $2^{32}$ bytes (4Gb) it is expected to only use one section in order to save the file content.}

	\subsubsection{Path}
	The path as a pointer to a null terminated string in the strings section. Used to identify relative extraction location and/or the virtual path of the asset in a rendering application.

	\subsubsection{Start}
	The index of the section to start reading the content of the file. In order to not over-read, refer to the size of the file as given in the file header.
 }

\subsection{Strings}
The strings section contains a list of null-terminated strings to be referenced by start offset from other sections (see \ref{ssec:Strings}).

\subsection{Metadata}
The metadata section contains generator specific information about the package. It is encoded as a Structured Data Section (see \ref{ssec:Structured}).
