\section{BPX Section Header Table}
The BPX Section Header Table is an array of data structures describing important information for each section present in the file. It is located after the BPX Main Header.\newline
The size of this array is determined by the SectionNum field in the BPX Main Header. The array is expected to be contiguous.\newline
Certain types of BPX expects to be able to index sections in this table using positions. For that reason, the array is expected to be 1-indexed to allow for a value of 0 to represent a null or non-existent section.\newline
This section is \textbf{not compressable}.

\subsection{BPX Section Header}
Below is a table describing the feilds to expect in a BPX Section Header:
\begin{center}
    {
        \rowcolors{2}
        {red!15}
        {blue!15}
        \begin{tabular}{|c|c|c|c|}
            \hline
            \textbf{Name} & \textbf{Type} & \textbf{Size} & \textbf{Notes} \\
    
            \hline\hline
            Pointer & Unsigned & 64 & Offset in bytes from the beginning of the file \\
            Size & Unsigned & 32 & Size in bytes of section \\
            Type & Unsigned & 8 & Type of Section \\
            Flags & Unsigned & 8 & Flags for the current section \\
            Reserved & Unspecified & 16 & Blank, always 0 \\
            \hline
        \end{tabular}
    }
\end{center}
\begin{center}
    \begin{bytefield}[bitwidth=0.73em]{64}
        \bitheader{0-63} \\
        \bitbox{64}{Pointer} \\
        \bitbox{32}{Size} & \bitbox{8}{Type} & \bitbox{8}{Flags} &     \bitbox{16}{Reserved}
    \end{bytefield}
\end{center}

\subsubsection{Size}
Size in bytes of the content for a given section excluding Section Header.

\subsubsection{Pointer}
Position in bytes in the file where the content for a given section should be found.

\subsubsection{Type}
An integer to represent the type of section.

\subsubsection{Flags}
Bit mask based flags (or flags together). Currently the only supported flags are:
\begin{center}
    {
        \rowcolors{2}
        {red!15}
        {blue!15}
        \begin{tabular}{|c|c|c|}
            \hline
            \textbf{Name} & \textbf{Value} & \textbf{Notes} \\
    
            \hline\hline
            Compressed & 0x2 & Indicates the section is compressed using the zlib \cite{zlib} algorithm \\
            \hline
        \end{tabular}
    }
\end{center}