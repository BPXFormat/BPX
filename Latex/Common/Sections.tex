\section{Standard Sections}

Additionally, any BPX type may use any of the following standard sections specifications. The count and usage requirements of such a section is still defined by the type of BPX. The type byte for this section is generally 0xFF (255).

\subsection{String section} \label{ssec:Strings}
A BPX may contain string section(s). A string section is a standardized way to store c-like strings in a BPX.

\subsubsection{Encoding}
The character encoding of strings in this section must be UTF-8.

\subsubsection{Structure}
The section is just a contiguous block of null-byte terminated UTF-8 encoded strings.\newline
Reading a string is done by taking its offset in the corresponding string section and starting to read characters until a null byte is found.\newline
Writing a string is done by taking the UTF-8 encoded bytes of this string and writing the bytes followed by a null byte at the end of the section.

\subsection{Structured Data Section} \label{ssec:Structured}
The BPX Structured Data format, also called BPXSD, borrows various concepts from JSON. All Structured Data sections begin by an Object (see \ref{sssec:Objects}). The type byte for this section is generally 0xFE (254).\newline
NOTE: This format is not exclusive to BPX: it may be used in other applications like network protocols.

\subsubsection{Type codes}
Each property is encoded with a Type byte also called type code.  Below is a table listing all type codes:

\bpxtable{c|c}{Type Code & Type Name}
{
	0 & NULL \\
	1 & Boolean \\
	2 & Uint8 \\
	3 & Uint16 \\
	4 & Uint32 \\
	5 & Uint64 \\
	6 & Int8 \\
	7 & Int16 \\
	8 & Int32 \\
	9 & Int64 \\
	10 & Float \\
	11 & Double \\
	12 & String \\
	13 & Array \\
	14 & Object \\
}

\subsubsection{Native types}
Below is a table showing encoding of native types. Native types are considerered fixed-size value types.

\bpxtable{c|c|C}{Type Name & Value size (bits) & Notes}
{
	NULL & 0 & A null value \\
	Boolean & 8 & A boolean value false (0), true (1) \\
	Uint8 & 8 & 8 bit unsigned integer \\
	Uint16 & 16 & 16 bit unsigned integer \\
	Uint32 & 32 & 32 bit unsigned integer \\
	Uint64 & 64 & 64 bit unsigned integer \\
	Int8 & 8 & 8 bit signed integer \\
	Int16 & 16 & 16 bit signed integer \\
	Int32 & 32 & 32 bit signed integer \\
	Int64 & 64 & 64 bit signed integer \\
	Float & 32 & 32 bit floating point number \\
	Double & 64 & 64 bit floating point number \\
}
For simplicity, booleans are encoded as 8 bit bytes.

\subsubsection{Strings}
Strings are encoded as a UTF-8 null-terminated contiguous block of bytes.

\subsubsection{Objects} \label{sssec:Objects}
An object is encoded as a contiguous list of properties each defined by a key-value pair. The first byte to be read of an object gives the number of properties in this object. The encoding of a property is as follows:

\bpxfieldtable
{
	Name & Unsigned & 64 & Hash of property name \\
	Type & Unsigned & 8 & Type of the property value \\
	Value & Unsigned & Depends on Type & Value of the property \\
}
\begin{center}
	\begin{bytefield}[bitwidth=0.73em]{64}
		\bitheader{0-63} \\
		\bitbox{64}{Name} \\
		\bitbox{8}{Type}
	\end{bytefield}
\end{center}

\subsubsection{Debugging}
As stated in the previous sub-section, the BPXSD Object stores hashes of the property names which means the original name is lost. To mitigate this issue, a BPXSD Object may contain a property, accessed by $Hash("\_\_debug\_\_")$, which stores debugging symbols. This property always points to an array of strings. In order to display property names, the implementation is responsible for finding in this debug symbol array the property name string such as $Hash(V_i) = P$ where $V$ is the array of debug symbols (i representing the index in that array) and P the property hash the implementation is searching for.

\subsubsection{Arrays}
An array is encoded as a contiguous list of values. The first byte to be read of an array gives the number of values in this array.

\bpxfieldtable
{
	Type & Unsigned & 8 & Type of the value \\
	Value & Unsigned & Depends on Type & Value \\
}
