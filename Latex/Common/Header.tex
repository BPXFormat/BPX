\section{BPX Main Header}
The BPX Main Header describes general information about the container and the contained data.\newline
This section is \textbf{not compressable}.\newline
Below is a table describing the different fields to be expected in the header:
\begin{center}
    {
        \rowcolors{2}
        {red!15}
        {blue!15}
        \begin{tabular}{|c|c|c|c|}
            \hline
            \textbf{Name} & \textbf{Type} & \textbf{Size} & \textbf{Notes} \\
    
            \hline\hline
            Signature & String & 24 & File signature; always "BPX" \\
            Type & Unsigned & 8 & Type of BPX \\
            Chksum & Unsigned & 32 & Checksum \\
            FileSize & Unsigned & 64 & Size of file after compression \\
            SectionNum & Unsigned & 32 & Number of sections \\
            Version & Unsigned & 32 & Version of format \\
            TypeExt & Unspecified & 128 & Extension space for various BPX types \\
            \hline
        \end{tabular}
    }
\end{center}
\begin{center}
    \begin{bytefield}[bitwidth=0.73em]{64}
        \bitheader{0-63} \\
        \bitbox{24}{Signature} & \bitbox{8}{Type} & \bitbox{32}{Chksum} \\
        \bitbox{64}{FileSize} \\
        \bitbox{32}{SectionNum} & \bitbox{32}{Version} \\
        \bitbox{64}{TypeExt} \\
        \bitbox{64}{TypeExt}
    \end{bytefield}
\end{center}

\subsection{Signature}
Three characters to describe the file when open in a text/hex editor

\subsection{Type}
The type of BPX. This is used to describe what type of sections to expect in the file.\newline
Currently only the following are supported:
\begin{itemize}
    \item 'T' for Texture
    \item 'M' for Model
    \item 'S' for Shader
    \item 'C' for Scene
    \item 'P' for Package
\end{itemize}
Here the characters between single quotes are to interpreted as their byte representation in ASCII encoding.

\subsection{Version}
Version of the file, the currently only available version of the format is 0.

\subsection{SectionNum}
Number of sections in the file.

\subsection{Chksum}
Checksum calculated with the help of a CRC32 algorithm. This should integrate both the header (except for the Chksum field) and the file content after compression.

\subsection{FileSize}
The file size field corresponds to the total size of the file minus the size of the header in bytes after compression; this is used as an additional security over the integrity of the file. It can also be used to check the remaining number of bytes in a network based streaming application.

\subsection{TypeExt}
The TypeExt field provides 64 bits of extension for different kind of BPX avoiding having to store or load additional sections in order to get usefull general information about a specific file reducing load time and memory complexity.\newline
All unused space in this field should be 0.
