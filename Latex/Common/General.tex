\section{General}
The format is based on the idea of a container with a reusable main header that can be extended for the different applications. While each BPX type defines it's own set of sections to load to comply to that given type, an implementation must be able to support any number of additional custom sections, even if those are not used in the application.\newline
The following assumptions are used:
\begin{itemize}
    \item A byte is assumed to be 8 bits.
    \item The sizes are expressed in bits in the entire document.
    \item The byte order in a BPX file should be \textbf{little endian} to match most of current hardware/software architectures, and then avoid a pre-processing step when loading the file.
    \item A transformation is stored in terms of float to match most current rendering APIs.
    \item The coordinate system used for this format is assumed to be right handed with the Z axis to be pointing upward.
\end{itemize}
Description of the coordinate system:
\begin{equation}
    Right =
    \begin{pmatrix}
        1 \\
        0 \\
        0
    \end{pmatrix}
    Forward =
    \begin{pmatrix}
        0 \\
        1 \\
        0
    \end{pmatrix}
    Up =
    \begin{pmatrix}
        0 \\
        0 \\
        1
    \end{pmatrix}
\end{equation}
The TypeExt section available in certain BPX types describes what fields to expect in the 128 bits of extension located in the BPX Main Header.\newline
If the TypeExt field isn't used by a specific BPX type then this field should be set to 0.

\section{Padding rules}
All data structures exposed in this format specification are padded to comply with major C/C++ compilers: Visual C++, GCC and Clang.

\section{Floating points}
All floating points in this format are stored in IEEE 754 format \cite{ieee754}.

\section{Hashing function}
Some types of BPX requires hashing for some strings. The string hash function to apply is defined by:

\begin{equation}
	h_X(0) = 5381
\end{equation}
\begin{equation}
	h_X(n + 1) = \left( \left( h_X(n) \times 2^5 \right) + h_X(n) \right) + X_{n + 1}
\end{equation}
where $X$ is a row vector containing the raw bytes of the string.