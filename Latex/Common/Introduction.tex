\section{Introduction}
When developing a 3D game engine there is often the problem of dealing with game modifications (mods). For example one might consider how easy it would be for the end player to add custom assets and/or change existing assets; whether the end player would need to agree against certain complex licenses or if it is even legally allowed to modify the content of certain asset files.\newline
Here the term asset refers to various types of files loaded by the game for a specific game world or for the entire game. These resources can be multimedia files, scripts, configurations and other formats the game engine could use.\newline
This format provides an alternative to existing asset formats geared towards the development of highly moddable, i.e. modify the game, add new assets, remove existing assets and/or change existing assets, and cross platform 3D games.\newline
The benefits are:
\begin{itemize}
    \item Open source with a permissive licence.\newline
    Redistribute source code, allow improvements of the format by others.
    \item DRM \cite{DRM} free.\newline
    Avoid licensing problems and restrictions. DRM stands for Digital Rights Management and is used to control how the user is allowed to interact with certain multimedia content. In moddable games DRM usually restricts the user from modifying existing assets.
    \item Flexible.\newline
    Designed for providing support for moddability. The nature of container allows support of all kinds of multimedia resources used in a typical real-time 3D application.
    \item Optimized for 3D APIs such as OpenGL \cite{OpenGL}, DirectX \cite{DirectX}, Vulkan \cite{Vulkan}\newline
    Certain types of BPX use some assumptions on the input data in order to reduce pre-process time when loading assets in a typical real time rendering engine.
    \item Cross platform.\newline
    Can store data for compatibility over multiple platforms within a single file due to its nature of containers.
\end{itemize}
